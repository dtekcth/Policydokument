\section{Sektionsoveraller}

\subsection{Bakgrund}
I syfte av att kunna representera sektionen vid arrangemang utan en egen fin overall finns det ett antal, men inte mindre än sju, så kallade ``sektionsoveraller'' (hädanefter kallade overallerna) som ägs av Datateknologsektionens styrelse (hädanefter kallat styret). Overallerna ska endast utnyttjas i sammanhäng där det är lämpligt.
Exempel på situationer då det är lämpligt att låna overallerna är arrangemang under Mottagningen då till exempel Nollan representerar sektionen eller vid aspaktiviteter då man vill låta asparna bära arbetskläder i sektionsfärgen.
Eftersom bärarna av dessa overaller representerar sektionen i sina aktiviteter får inte overallerna användas för arrangemang som negativt påverkar Datateknologsektionens rykte eller går emot våra stadgar.

Denna policy existerar i syfte för att säkerställa overallernas skick genom några få och enkla regler.
Vid utlånande av ovan nämnda overaller ska ett skriftligt kontrakt formuleras mellan styret och lånande parten. Tidsperioden för utlånandet ska tydligt framgå i kontraktet. I och med utlånandet av overallerna erhåller den utlånande parten fullt ansvar för overallen/erna. För att nedanstående regler ska följas betalas en depositionsavgift på 100kr in till styret. Depositionen återfås i fullo vid inlämnad overall om nedanstående regler följts.

\subsection{Regler}
\begin{itemize}
\item Overallen/erna ska återlämnas i samma skick som den utlånades.
  \begin{itemize}
  \item Om overallen ej kan återställas, men skadan inte är allvarlig nog att det är störande kommer inte depositionsavgiften att återbetalas.
  \item Om overallen ej kan återställas och skadan är tillräckligt allvarlig att det förstör overallen så ska den lånande parten bekosta en ersättningsoverall.
  \end{itemize}
\item Man får inte bära overallen/erna på ett sätt som strider mot Chalmers eller Datateknologsektionens stadga.
\item Overallen/overallerna ska återlämnas vid avsagd tid.
\end{itemize}

\vspace{1cm}

\emph{Sektionsstyrelsen 2015}