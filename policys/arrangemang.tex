\section{Arrangemangspolicy}
\subsection{Introduktion}
Denna policy gäller alla arrangemang i Datateknologsektionens regi, samt övriga arrangemang i de lokaler som disponeras av Datateknologsektionen.

\subsection{Planering}
Alla arrangemang utan speciell målgrupp skall vara inkluderande för alla medlemmar. Ett exempel på målgrupp kan vara nyantagna under mottagningen.

\subsection{Kommunikation}
Alla arrangemang riktade mot sektionens medlemmar ska informeras om via \emph{minst} ett av följande:
\begin{itemize}
    \item sektionens nyhetsbrev
    \item anslagstavla i Basen
\end{itemize}
All nödvändig information om arrangemanget ska finnas tillgänglig på engelska. PR skall informera om ifall svenskkunskap är rekommenderat för besökare, exempelvis på sittiningar ''toastade'' på svenska.

\subsection{Hållbarhet}
Alla arrangemang ska sikta mot att minimera sin miljö- och klimatpåverkan.

\begin{itemize}
    \item I största utsträckning skall återanvändbara tallrikar, bestick och glas användas.
    \item Alla rester från arrangemang skall källsorteras.
\end{itemize}

\subsection{Alkohol och droger}
Eventuell servering av alkohol skall ske med måtta. Då alkohol serveras skall alltid fullvärdiga alkoholfria alternativ, såsom alkoholfri öl och cider, finnas tillgängliga.
