% \documentclass{dtek}
% \usepackage[utf8]{inputenc}
% \usepackage{todonotes}

% \title{Datateknologsektions policy för personuppgifter}
% \date{\today}

% \usepackage{natbib}
% \usepackage{graphicx}


% \begin{document}

% \maketitle
% \makeheadfoot

\section{Personuppgiftspolicy}

Använd den här policyn som en guide för att hjälpa dig i ditt GDPR-arbete. Om du har några frågor eller funderingar, kontakta styret: \href{mailto:styret@dtek.se}{styret@dtek.se}.

Den här policyn skapades 2018 som ett led i arbetet med att göra sektionen, dess kommittéer och intresseföreningar kompatibla med GDPR och ska om allt fungerar som det är tänkt uppdateras löpande.

% \tableofcontents

\subsection{Ändringslogg}

Här kommer ändringar med datum finnas när dokumentet uppdateras.

\subsection{Om GDPR och ideellt engagemang}

Det är viktigt att det är lätt och roligt att engagera sig ideellt på datateknologsektionen, även när vi behöver hålla oss till gällande lagar och regler. Därför försöker vi hålla administrationen kring GDPR till ett minimum. Följande behöver dock utföras för att vi ska följa lagen:

\subsection{Vid inval}

Den nyinvalde ska så snart som möjligt skriva på ett kontrakt för att bli personuppgiftsbehandlare, vilket innebär att den får samla in och hantera personuppgifter för Datateknologsektionen.

Den nyinvalde behöver också skriva på ett kontrakt där den godkänner att dess personuppgifter lagras och används på det sättet som krävs för kommitténs/intresseföreningens verksamhet.

\subsection{Vid inaktivitet}

Om en person slutar vara aktiv i en kommitté eller intresseförening förlorar den åtkomst till kommitténs personuppgifter tills den blir aktiv igen, detta för att minimera antalet personer som kan drälla 
% \todo{vill vi använda sådant språk? [+] Mer lätttilgängligt [-] mindre formellt} bort personuppgifter.

\subsection{Tredjepartsåtkomst}

Utomstående kan få åtkomst till personuppgifter om det föreligger särskilda skäl, förutsatt att detta godkänns av personuppgiftsansvarige i Styrelsen. Vid sådana omständigheter ska personen i fråga även underteckna ovan nämnda kontrakt. 
% \todo{personuppgiftsbiträde, vilka tänker vi här. Och hur ser processen ut för personuppgiftsansvarige?}

\subsection{Dokumentation}
    
Styret ska upprätthålla en lista över alla personer med åtkomst till personuppgifter, samt i vilka sammanhang de kommer åt dessa personuppgifter. Exempel på sammanhang är ''Phixare i D6''.
% \todo{utöver sammanhang borde det klargöras mot vilka personuppgiftsregister de har access}
% \todo{Styrelsen bör ha en lista över personuppgiftsregister också}

\subsection{Behandling av personuppgifter}

GDPR är hårdare kring lagring av personuppgifter än den tidigare personuppgiftslagen (PUL). Nedan följer praktiska riktlinjer för hur personuppgifter ska hanteras med nya lagstiftningen.

\subsection{Opt-in -- aktivt medgivande}
\label{ss:optin}

GDPR kräver opt-in
% \footnote{Engelska är ett fantastiskt språk: \href{https://en.wiktionary.org/wiki/opt\#English}{https://en.wiktionary.org/wiki/opt\#English} } d.v.s. att varje person vars personuppgifter ska lagras ger aktivt samtycke. Det innebär att de vars personuppgifter vi lagrar måste ge aktivt samtycke \textit{innan} vi lagrar deras personuppgifter och att hen kan anses förstå innebörden av datalagringen\todo{kan formuleras bättre, men essensen är att vi inte kan säga "Jag godkänner att ge bort all min data till vad som hellst"}. Några exempel:
\begin{itemize}
    \item På anmälan till ett evenemang behöver det finnas en obligatorisk kryssruta där den som anmäler sig godkänner att dess personuppgifter sparas för att arret ska kunna genomföras. Exempel på text:\\
    
    \noindent\fbox{%
        \parbox{\textwidth}{%
        ``För att arrangemanget ska gå att planera och genomföra behöver vi spara dina ovan angivna uppgifter under en begränsad tid. Dina personuppgifter kan komma att delas med tredje part, t.ex. puffar eller samarrangörer, om det behövs för arrangemanget. Strax efter arrangemanget kommer uppgifterna att raderas.''
        }%
    }
    
    \item På puff-formulär då man vill spara puffarnas uppgifter längre än till strax efter eventet då man vill bjuda på tackkalas eller be om hjälp vid senare tillfälle kan följande stå:\\
    
    \noindent\fbox{%
    \parbox{\textwidth}{%
    ``Dina uppgifter kommer att sparas för att vi ska kunna kontakta dig även vid senare tillfälle, t.ex. om vi behöver hjälp vid något annat arrangemang eller på något sätt vill tacka dig för din insats. Dina uppgifter kommer raderas vid slutet av vårt verksamhetsår.''
        }
    }
\end{itemize}

\subsection{Tidsbestämt}

Uppgifter får inte sparas längre än nödvändigt. Några exempel:

\begin{itemize}
    \item Anmälningslistor till sittningar ska raderas senast två veckor efter sittningen.
    \item Pufflistor får sparas till två veckor efter tackkalaset.
\end{itemize}

Väl värt att komma ihåg är att den vars personuppgifter sparas behöver godkänna hur länge de sparas, se~``\nameref{ss:optin}'' för detaljer kring explicit samtycke och tidsbegränsning.

Statistik får sparas, t.ex. hur många som var på sittningen, hur många av dem som var vegetarianer o.s.v. så länge detta inte går att koppla till en enskild individ. Allergier eller intoleranser som få individer har måste därför slängas.

\subsection{Specificitet}

En personuppgift får bara användas till det som den som äger personuppgiften har godkänt att den får användas till.

Antag att det finns ett godkännande för att spara e-postadresser för att kunna kontakta personer. Då är det endast okej för detta syfte och man får då \textit{inte} använda dessa uppgifter för att skapa en mailinglista till alla som går på data eller dela uppgifterna till tredje part.

\subsection{Spara alltid hela namnet}

Fullständigt namn ska alltid sparas tillsammans med andra personuppgifter för en person så att personuppgiftsansvarige har en rimlig möjlighet att t.ex. ta bort en persons personuppgifter om den vill bli bortglömd.

Det kan vara svårt att få deltagare att fylla i sina namn. Därför rekommenderar vi att fullständigt namn och smeknamn samlas in som separata fält för att uppmuntra de som fyller i listan eller formuläret att ge användbar information.

\subsection{G-Suite och hantering av data}

Vi har valt att använda G-Suite för att lagra vår data. Detta för att den t.ex. har Google Docs och andra användbara verktyg och fungerar väl med organisationer.

\subsection{Användarstruktur}

Varje användare ligger under en organisation som motsvarar en kommitté eller förening. För poster som ingår i flera kommittéer ligger dessa under dess primära kommitté/förening, till exempel ligger ordförande i Delta primärt i Delta men även i Styret.

Användare läggs därefter till i grupper. Varje grupp har en e-postadress som man kan nå gruppen på. I dessa grupper ska alla i kommittén/föreningen finnas med. Till exempel ska sektionskassören stå under organisationsenheten Presidiet men ska finnas med i både gruppen Presidiet \textit{och} dBus.

Kontakta personuppgiftsansvarig i Styret för att lägga till nya organisationer och/eller grupper.

\subsection{Teamdrive}

Team Drive är som en vanlig myDrive men det är gruppen och inte individerna som äger innehållet. Alla kommittéer och föreningar får ha en och endast en TeamDrive som de ärver från tidigare år. För dessa gäller:
% \todo{undantag temporära}

\begin{enumerate}
    \item Högst upp i mapphierarkin bör det finnas en mapp för varje år, t.ex. ``2017'' och ``2018''.\\
    Om kommitténs arbete förenklas om man frångår ovanstående är det okej. Till exempel kan inventeringslistor kan vara bra att ha på högsta nivån.
    \item Alla mappar och dokument döps så att en utomstående förstår vad det handlar om. Olämpliga namn kan vara ``Supa satan mapp'' och ``Lite roligt bös''. Döp dessa istället till ``Postom'' och ``Förslag på arr'' så följer ni policyn och gör livet lättare för era efterträdare.
    \item Alla dokument och mappar som innehåller personuppgifter ska ha ett namn med suffixet ``[PU]'' (PersonUppgifter). T.ex. mappen ``Möten[PU]'' eller filen ``Kontaktuppgifter sittande[PU]''.
\end{enumerate}

%Om ni lyckas med ovanstående kan ni gå till studiesekreteraren och tillgodoräkna er 3,75hp informationsdesign.
%lol

Det är inte allt för sällan som man har ett gemensamt arrangemang med andra kommittéer eller puffar som behöver ta del av ett eller flera dokument. Eftersom TeamDrive inte har stöd att dela hela mappar utan bara filer gäller följande:

\begin{enumerate}
    \item Om det är puffar som klarar sig med enstaka dokument, dela då endast dessa.
    \item Om det är ett samarr och utomstående/andra kommittéer behöver tillgång till alla filer är det okej att skapa en temporär Team Drive dit man bjuder in alla arrangörer som behöver tillgång till filerna. Denna Team Driven ska raderas alternativt ska allt innehåll flyttas efter arrangemanget om inte speciella omständigheter råder. Personuppgiftsansvarige beslutar om det råder speciella omtändigheter. Ovanstående regler gäller även för dessa.
\end{enumerate}

\paragraph{Tips:} Om det är en hel grupp som ska ha tillgång till en drive räcker det med att gruppen bjuds in för att alla ska få tillgång.

\paragraph{OBS!} Tänk på att ni måste ha godkännande att dela till tredje part, se Opt-in ovan. Intresseföreningar och kommittéer på data räknas inte som tredje part eftersom de är underorganisationer till datateknologsektionen. %\todo[inline]{Är det verkligen så? Enl. opt-in ovan ger vi bara medgivande till att våra personuppgifter används för en väldigt specifik sak, inget annat. I så fall behöver avtalen/texten vid kryssrutorna innehålla att data kan komma att delas till andra delar av data, eller har jag fått något om bakfoten? //A}
%\todo[inline]{Lagen ser ingen skillnad mellan data som data oavsett delta eller d6 osv eftersom vi ligger under samma organisation, så är det ett samarr mellan dataföreningar så delar vi inte till 3:e part. Men är det samarr med annan sektion/om vi vill ge datan till puffar så är detta tredje part. Min tolkning iaf /Anna}

\subsection{Egen drive}
Med ditt inlogg följer en egen drivemapp. Denna ska alltid, med undantag för SAMO, vara fri från personuppgifter. I övrigt bör strukturen vara som för en team-drive:

\begin{enumerate}
    \item En mapp för varje år. T.ex. ``2017'', ``2018''. 
    \item Det är okej att ha en mapp för alla år tidigare. T.ex. ``1337--2016''
\end{enumerate}

Vi rekommenderar att man följer ovanstående struktur för att göra det lätt för nya att snabbt få en överblick över vilka dokument som finns och var de ligger, men om den strukturen inte fungerar, använd en som fungerar bättre för er.

\subsubsection{SAMO}

SAMO är den enda som får spara egna uppgifter i sin egen mapp då hen hanterar känsliga uppgifter som inte får användas av andra. Eftersom SAMO får ha personuppgifter i sin drive rekommenderas det att strukturen följer den som för en Team-drive.

Om någon ber att bli bortglömd ska SAMO vara personuppgiftsansvarige behjälplig.

\subsection{GMail}

Använd enbart e-postadresserna du har genom sektionen till sektionsarbete, det är alltså både otillåtet och dumt att använda den för privat bruk då du lämnar över adressen till din efterträdare efter ditt verksamhetsår.

All e-post raderas automatiskt efter 1337 dagar.
%Ja, 1337 är på riktigt. det var fun /Anna

E-post får ej vidarebefordras till privata e-postadresser.

\paragraph{Tips:}  Sätt en signatur som automatiskt infogas i slutet av dina mejl. Förslag:

%\todo{Denna lär nog i det generella fallet vara satt av en patet, bör skrivas som att det ska uppdateras kanske?}
% Bara ett förslag. Står redan i "slutet av ditt verksamhetsår".

\noindent\fbox{%
    \parbox{\textwidth}{%
        Vänliga hälsningar,\\
        <Förnamn Efternamn>\\
        <Förening> <Post> 20xx/20xx\\
        Datateknologsektionen Chalmers Studentkår\\
        www.dtek.se<http://www.dtek.se/>
    }
}

Går även att infoga bilder. Vill man använda sektionens logga kan man hitta den här:\\
\url{https://avatars1.githubusercontent.com/u/2522755?s=280\&v=4}

% \todo{Borde finnas något om att inte registrera sig på ställen (typ Twitter, LinkedIn m.m.) med sin e-post om det inte är strikt postrelaterat.}

% \subsection{E-postlistor}

% \todo[inline]{Lägg till en guide på hur man bjuder in folk till e-postlistor, särskilt m.a.p patetlistor. \\ /A}

\subsection{Kalender}

Man får gärna skapa en eller flera kalendrar. Viktigt är att tänka på sätta rätt åtkomsträttigheter och dela med rätt personer. Till exempel kan en bokningskalender som ``databussen'' eller ``Basenpax'', ha följande inställningar:
\begin{itemize}
    \item Tillgänglig för alla: se endast ledig/upptagen (dölj uppgifter).
    \item Delad med: dbus@dtek.se respektive alla@dtek.se. Rätt att göra ändringar och hantera delning.
\end{itemize}

En intern kalender bör endast vara delad med kommittén och en kalender som delas med utomstående ska inte innehålla personuppgifter som de kan se. 

Man kan lägga till hela grupper i en kalender, men då måste varje medlem lägga till kalendern via en länk som skickas till mejlen. Några bra kalendrar som finns som kan vara bra för alla datateknologer är:

\begin{itemize}
    \item \textbf{Dtek-kalendern:} \\
    \url{https://calendar.google.com/calendar/embed?src=dtek.se_0tavt7qtqphv86l4stb0aj3j88\%40group.calendar.google.com&ctz=Europe\%2FStockholm}
    \item \textbf{Basenpaxningar:} \\
    \url{https://calendar.google.com/calendar/embed?src=dtek.se_b3sv1v3upmtjquppg10hhe59e0\%40group.calendar.google.com&ctz=Europe\%2FStockholm}
    \item \textbf{Databussen:} \\
    \url{https://calendar.google.com/calendar/embed?src=dtek.se_69sdfhe5527mh3u9tk9146imak\%40group.calendar.google.com\&ctz=Europe\%2FStockholm}
\end{itemize}

\subsection{Administration}

Administrera G-Suite genom att logga in på \url{https://admin.google.com/dtek.se/AdminHome} med användarnamn och lösenord. Ordförande i varje kommitté/intresseförening är administratör och har följande rättigheter:

\paragraph{Användarhanteringsadmin}
Kan utföra alla åtgärder på användare som inte är administratörer. Den här administratören kan utföra följande uppgifter både från administratörskonsolen och via Admin API:

\begin{itemize}
    \item Visa användarprofiler och din organisationsstruktur
    \item Läsa organisationsenheter
    \item Skapa och ta bort användarkonton\footnote{\label{note1}Gäller endast för användare som inte är administratörer. Denna administratör kan inte tilldela administratörsbehörighet, återställa ett administratörslösenord eller göra andra ändringar av ett administratörskonto. Det är bara avancerade administratörer som kan göra detta.}. OBS, det ska endast finnas ett användarkonto för varje post.
    \item Byta namn på användare och ändra lösenord$^{\ref{note1}}$.
    \item Hantera en användares individuella säkerhetsinställningar$^{\ref{note1}}$.
    \item Utföra dessa övriga uppgifter för användarhantering$^{\ref{note1}}$.
\end{itemize}


\textbf{Supportavdelningsadministratör}

Kan återställa lösenord för användare som inte är administratörer, både på administratörskonsolen och via Admin API. Den här administratören kan också visa användarprofiler och din organisationsstruktur. Denna administratör kan bara läsa organisationsenheter.

\subsubsection{Hantering av konton}

Ifall en post byter namn, tas bort eller skapas gäller följande:
\begin{itemize}
    \item ``Namnet'' ska följa formatet Förnamn = post, Efternamn = Förening.
    \item Den primära epost-adressen ska vara post@föreningen.dtek.se. Ordförande förkortas till ordf, kassör till kass.
\end{itemize}

Profilbilder kan andras av ordföranden i admininterfacet. Undvik bilder på personer. Använd mycket hellre en logga, illustaration eller bild på något som inte är personer. 

\subsection{I början av verksamhetsåret}

\begin{enumerate}
    \item Byt lösenord. Ordförande ser till att detta blir gjort. Detta kan göras genom att gå in på varje användare, t.ex.  Användare > Ordförande DAG, och där använda funktionen ''Återställ lösenord''.
    \item Uppdatera signatur till mejlen (om detta fanns sen tidigare). 
\end{enumerate}

\subsection{I slutet av verksamhetsåret}
\begin{enumerate}
    \item Storrensning i mappen. Allt som inte kommer vara till användning ska raderas.
    \item Sök efter suffixet ``[PU]'' och rensa allt. Om något ska sparas görs detta i samråd med personuppgiftsansvarige.
    \item Rensa mejlen.
\end{enumerate}

\subsection{Handlingsplan vid dataläcka}

Vid minsta misstanke om dataläcka, kontakta omedelbart personuppgiftsansvarig i Styret: \mbox{\href{mailto:styret@dtek.se}{styret@dtek.se}.}

%\bibliographystyle{plain}
%\bibliography{references}
% \end{document}