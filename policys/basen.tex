\section{Basenpolicy}
\subsection{Introduktion}
Denna policy gäller alla som ämnar nyttja Datateknologsektionens lokal Basen.
\subsection{Nyttjande av lokalen}
\subsubsection{Rättigheter}
Alla medlemmar i Datateknologsektionen har fri tillgång till Basen. 

\subsubsection{Skyldigheter}
\label{sec:basen:skyldigheter}
Medlem skall
\begin{itemize}
    \item följa sektionens och kårens styrdokument, samt svensk lag.
    \item städa efter sig själv.
    \item ej störa andra.
    \item ej avlägsna objekt från Basen utan styrelsens godkännande.
    \item ej förvara skrymmande saker i Basen eller dess närhet.
    \item medverka i Nollstäd.
\end{itemize}
Medlem ansvarar för sina gäster.

\subsubsection{Undantag}
\begin{itemize}
    \item Medlem som inte följer skyldigheterna i \ref{sec:basen:skyldigheter} kan fråntas rätten att vistas i lokalen.
    \item Vid arrangemang begränsas ytan för gemene teknolog till köket. Vid vissa arrangemang är även vistelse i köket förbjudet, exempelvis under serveringstillstånd.
    \item Under längre skoluppehåll är Basen avstängd, då resurser inte finns för att städa den.
    \item Delar av, eller hela, Basen kan efter styrelsebeslut stängas av av andra anledningar.
\end{itemize}

\subsection{Arrangemang i lokalen}
Alla arrangemang i Basen skall följa Arrangemangspolicyn samt nedanstående regler.
\subsubsection{Regler för arrangemang}
\begin{itemize}
    \item Ett arrangemang ska alltid anmälas tre dagar innan till styrelsen
    \item Man ska alltid fylla i och städa enligt de arr-städ lappar som finns att tillgå i Basens städskrubb.
    \item Man får bara bruka alkohol i Basen mellan 17:00 och 03:00 Söndag - Torsdag
    \item Man får bara bruka alkohol i Basen mellan 17:00 och 05:00 Fredag - Lördag
    \item Man måste anmäla till styrelsen ifall det är mer än 4 personer som brukar alkohol i Basen
    \item Det är bara sektionsmedlemmar som kan vara ansvariga
    \item Ansvariga ska alltid hålla sig vid sina sinnens fulla bruk
    \item Man får aldrig bruka alkohol i Basenköket
    \item Man får aldrig vara fler än 70 personer i lokalen
    \item Man får aldrig bryta mot några av Chalmers Studentkårs regler
\end{itemize}
\subsubsection{Paxa basen för arr}
För att utföra ett planerat arr i Basen så ska man anmäla detta till styrelsen minst 3 dagar innan arrangemanget. I en anmälan så ska man specifiera: Vad för typ av arr det handlar om, ungefärlig mängd personer, tid och datum, och 1-2 personer som bär ansvar samt
andra medarrangörer.

Efter godkännade från styrelsen så ska arrangemanget aktivitetanmälas på studentportalen senast 2 dagar innan arrangemanget. Där ska även den ansvariga skrivas upp.

Under ett arrangemang så ska reglerna för uppförande följas och arrangörerna ser till att dessa följs av alla närvarande. Arrangörerna har rätt till att avvisa folk från lokalen som
inte följer dessa regler. Datateknologer har dock alltid rätt att få tillgång till Basens kök.

Efter avslutat arrangemang så ska städning enligt specificering nedan följas. Vid misskötsel eller ej godkänt städ kommer ansvariga samt medarrangörer bli kontaktade med rimlig efterföljd som bestäms av styrelsen.

\subsubsection{Oanmälda aktiviteter}
Ett arrangemang måste anmälas till styrelsen enligt ovanstående regler ifall fler än 4 personer
ska vistas i basen i samband med alkoholhaltiga drycker. Städning enligt specificering nedan ska alltid följas.
Ifall ett sällskap vistas i Basen och fler personer ansluter sig till sällskapet så den totala
summan människor blir fler än 4 så ska detta anmälas till styrelsen. Ifall det tillkommer ett
till sällskap så den totala summan människor blir över 4 så ska detta anmälas till styrelsen.

Anmälning till styrelsen görs via ett mail till styrelsen där man anmäler ansvariga personer.

\subsubsection{Städ}
Om man har arrangerat i Basen så ska det städas enligt de instruktioner som finns på arr-städ lapparna hängandes i Basens städskrubb. Städningen ska vara avklarade innan 08:00 följande morgon. Om det inte är städat innan dess eller om DRust alternativt Styret anser att du inte har städat eller fyllt i hela listan kommer städningen inte att godkännas.

%\begin{itemize}
%    \item Släng burkar/flaskor och skräp i sopkorgen
%    \item Torka av bord, stolar och soffor
%    \item Sopa golvet
%    \item Svabba golvet
%    \item Se till att toaletterna är rena från \textit{olyckor} samt att inte papper o. dyl. ligger på golvet.
%    \item Släng sopsäckar med burkar i soprummet
%    \item Se till att inga spår finns av fest
%    \item Se till att skrubben hålls iordning
%\end{itemize}
\subsection{Åtgärder}
Om några regler bryts kommer åtgärder åtas. Ärendet kommer tas upp på ett styrelsemöte och styrelsen har rätt att bestämma vilka åtgärder som kommer tas. Exempel på åtgärder är att personen i fråga får ett rött kort; det vill säga indragen tillgång till Basen och alla arrangemang i den.